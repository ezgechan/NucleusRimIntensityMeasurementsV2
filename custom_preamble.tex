%% This file is for adding custom environments!
%% please add / delete / modify as you like. 

%% a block for excercies
\newenvironment{indentexercise}[1]%
{{\setlength{\leftmargin}{2em}}%
\textbf{Exercise \thesubsection-#1}%
\begin{list}{}% 
	\item%
}
{\end{list}}

%% Indenting for ImageJ commands in a single line. 
\newenvironment{indentFiji}%
{\begin{list}{}%
         {\setlength{\leftmargin}{1em}}%
         \item[]%
}
{\end{list}}

%% indenting for what ever command in a single line. 
\newenvironment{indentCom}%
{\begin{list}{}%
         {\setlength{\leftmargin}{1em}}%
         \item[]%
}
{\end{list}}

%%% command for imageJ menu tree, in-line
\newcommand{\ijmenu}[1]{\texttt{\small#1}}

%%% command for inline command
\newcommand{\ilcom}[1]{\texttt{\small#1}}

%%% a quick command for making tab space
 \newcommand{\tab}{\hspace*{3em}}

%%% a quick command for horizontal line. 
\newcommand{\HRule}{\rule{\linewidth}{0.5mm}}

%% textcomp provides extra control sequences for accessing text symbols:
\usepackage{textcomp}
\newcommand*{\micro}{\textmu}
%% Here, we define the \micro command to print a text "mu".
%% "\newcommand" returns an error if "\micro" is already defined.


%%%%%%%%  Source Code Matters %%%%%%%%
%% please keep settings for "listings" not too much changes: but the "language" should be changed to the code that you are mainly using.  

\usepackage{alltt}

% packge for codes
\usepackage{listings}
%\usepackage{listingsutf8}
\lstset{ %
language=matlab,                % could choose the language of the code, but we go for black and white, no syntax highlighting defined. 
%basicstyle=\footnotesize,       % the size of the fonts that are used for the code
basicstyle=\small\ttfamily, % same as above, but use typewriter
numbers=left,                   % where to put the line-numbers
numberstyle=\footnotesize,      % the size of the fonts that are used for the line-numbers
stepnumber=1,                   % the step between two line-numbers. If it's 1 each line 
                                % will be numbered
numbersep=5pt,                  % how far the line-numbers are from the code
backgroundcolor=\color{gray09},  % choose the background color. You must add \usepackage{color}
keywordstyle=\color{blue}, 	%added
showspaces=false,               % show spaces adding particular underscores
showstringspaces=false,         % underline spaces within strings
showtabs=false,                 % show tabs within strings adding particular underscores
%frame=single,                   % adds a frame around the code
%frame=trBL,
tabsize=2,                      % sets default tabsize to 2 spaces
captionpos=b,                   % sets the caption-position to bottom
breaklines=true,                % sets automatic line breaking
%breakatwhitespace=false,        % sets if automatic breaks should only happen at whitespace
title=\lstname,                 % show the filename of files included with \lstinputlisting;
                                % also try caption instead of title
escapeinside={\%*}{*)},         % if you want to add a comment within your code
morekeywords={*,...},            % if you want to add more keywords to the set
morecomment=[l]{//},
morecomment=[s]{/*}{*/},
morestring=[b]",
%aboveskip={7.2pt}	%supposed to be the space above llisting but dows not work. 
%belowskip={7.2pt}
}